%! Author = michal
%! Date = 4/21/20

\section{Introduction}
\label{sec:introduction}

Transport layer of ISO/OSI model contains two most significant protocols in terms of
Internet communication -- TCP and UDP~.

TCP is a connection based protocol which handles congestion control, sends confirmations
when message is received or cannot be received and guarantees proper order of sending
packets.
Because TCP is a point to point protocol we are not allowed to realize multicast
transmission using it.
The biggest drawback of this protocol is its heaviness.
Connection setup requires so called three way handshake which takes some time.
Lack of any built-in security mechanisms forces us to use dedicated to this purpose
TLS protocol which introduces its own connection handshake.
This causes that connection setup becomes even less efficient.
On the other hand there is a very lightweight, connectionless protocol called UDP~.
It does not guarantee proper order of sending packets.
There are not also any confirmations or warnings sent in case of communication errors.
Everything is performance oriented.
UDP provides also multicast mechanism.
According to this UDP is a good choice when we need very fast and lightweight
communication but we can lose some packets.
A good example of UDP destination is a video streaming.

We can see that these two protocols and their destinations are strongly opposite.
And here comes QUIC -- new transport layer protocol based on UDP which is intended
to replace TCP providing higher performance and even better reliability.
First of all QUIC is user level protocol not system level one.
It means that we can implement it without modifying operating systems kernels.
Secondly QUIC is a secure protocol.
It ensure data encryption and authentication which are handled by some other
protocols like TLS~.
QUIC provides support for these protocols.
Figure~\ref{fig:http-req-comparison} presents comparison between HTTP request
made in TCP and QUIC protocols.
Typical QUIC handshake takes a single round-trip between hosts when TCP one requires
two round-trips.

\begin{figure}
    \centering
    \begin{subfigure}{.5\textwidth}
        \begin{sequencediagram}
            \newinst{client}{Client}
            \newinst[3]{server}{Server}
            \mess{client}{TCP SYN}{server}
            \mess{server}{TCP SYN + ACK}{client}
            \mess{client}{TCP ACK}{server}
            \postlevel
            \mess{client}{TLS ClientHello}{server}
            \mess{server}{TLS ServerHello}{client}
            \mess{client}{TLS Finished}{server}
            \postlevel
            \mess{client}{HTTP REQ}{server}
            \mess{server}{HTTP RES}{client}
        \end{sequencediagram}
        \caption{HTTP request in TCP}
        \label{http-req-tcp}
    \end{subfigure}%
    \begin{subfigure}{.5\textwidth}
        \begin{sequencediagram}
            \newinst{client}{Client}
            \newinst[3]{server}{Server}
            \mess{client}{QUIC}{server}
            \mess{server}{QUIC}{client}
            \mess{client}{QUIC}{server}
            \postlevel
            \mess{client}{HTTP REQ}{server}
            \mess{server}{HTTP RES}{client}
        \end{sequencediagram}
        \caption{HTTP request in QUIC}
        \label{http-req-quic}
    \end{subfigure}
    \caption{HTTP request comparison}
    \label{fig:http-req-comparison}
\end{figure}

Another enhancement in QUIC is resistance for network changes.
User can change its Wi-Fi network without performing a new client-server handshake.
This is achieved by special ID's QUIC associates with each connection.

Results section describes performance differences between communication in TCP and QUIC
protocols including connection time setup and reliability.
